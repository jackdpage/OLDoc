\documentclass[a4paper]{article}
\usepackage{hyperref}
\begin{document}
\title{Pylux User Manual \\ \large{for Pylux v0.1-alpha2}}
\author{J. Page}
\date{2015}
\maketitle
\vspace{1.0pt}
Copyright (C)  2015 Jack Page \\
Permission is granted to copy, distribute and/or modify this document
in source form (LaTeX) or compiled form (PDF, PostScript, etc.), including 
for commercial use, provided that this copyright notice is retained and that 
you grant the same freedoms to any recipients of your modifications.
\tableofcontents
\newpage
\section{Introduction}
Pylux is a program written in Python for the manipulation of OpenLighting Plot
documents. Pylux currently allows for the basic editing of the OpenLighting
Plot XML files, including referencing OpenLighting Fixture files to obtain
additional data.

In the future, Pylux will be extended with modules that allow for the exporting
of documentation in the \LaTeX{} format, and creating lighting plots in SVG 
format.

Bugs and feature requests should be submitted to 
\url{https://github.com/jackdpage/pylux/issues}.

\section{Command-line Options}
Pylux is invoked simply by running the \texttt{plotter.py} program. Windows 
operating systems should automatically associate file with the Python 
interpreter. The program contains a shebang line for compatibility with UNIX
systems. 

\paragraph{\texttt{--help}, \texttt{-h}}
Display the usage message of the program then exit.

\paragraph{\texttt{--version}, \texttt{-v}}
Display the version number of the program then exit.

\paragraph{\texttt{--file \textit{FILE}}, \texttt{-f \textit{FILE}}} 
Load the file with path \texttt{\textit{FILE}} as the plot file on launch.

\paragraph{\texttt{--gui}, \texttt{-g}}
Launch Pylux with the graphical user interface. This is not yet implemented. 
Omitting this flag launches Pylux with its standard CLI interface.

\section{The Plot File}
Pylux uses the Pylux plot format as its document format. This format 
specifies three main components of an XML document: the metadata section, the 
fixtures section and the registries section. 

\subsection{The Metadata Section}
The metadata section of the plot file contains any additional information 
that does not fit inside the other sections. The information in this section 
is often called upon to produce output.

\subsubsection{Standard Metadata Tags}
Whilst you can use any name for a tag you wish, there are some standard ones 
which are used by Pylux and its bundled extensions.

\paragraph{\texttt{production}}
The name of the production for which the lighting documentation is being 
produced, e.g. 'Romeo and Juliet'. Used by: \texttt{texlux}.

\paragraph{\texttt{designer}}
The name of the lighting designer for this production. Used by: 
\texttt{texlux}.

\paragraph{\texttt{board\_operator}}
The name of the person operating the main lighting board for this production.
Used by: \texttt{texlux}.

\paragraph{\texttt{spot\_operator}}
The name of the person operating the primary followspot for this production.
Used by: \texttt{texlux}.

\paragraph{\texttt{director}}
The name of the director of the production. Used by: \texttt{texlux}.

\paragraph{\texttt{venue}}
The location at which the production is taking place. Used by: 
\texttt{texlux}.

\subsection{The Fixtures Section}
The fixtures section consists of a base \texttt{fixtures} element and 
multiple \texttt{fixture} child elements, each of which describes one fixture 
within the plot file. Each fixture is assigned a UUID on creation, which 
never changes. Each fixture also has children which represent its other data.
Some of these children will have been imported from the fixture template 
file on the fixture's creation but others you will have to define yourself.

\subsubsection{Standard Fixture Data Tags}

\paragraph{\texttt{dmx\_functions}}
This is the parent of a list of empty elements, each of which represents a 
function that the fixture has that requires the use of a DMX channel. For 
example, traditional fixtures will have an \texttt{intensity} function 
whilst modern LED fixtures may have \texttt{colour} and \texttt{mode} 
functions. Used by: \texttt{plotter}.

\paragraph{\texttt{dmx\_channels}}
The number of DMX channels that a fixture needs. This is automatically 
calculated from the \texttt{dmx\_functions} tag, so should not be changed 
manually. Used by: \texttt{plotter}.

\paragraph{\texttt{dmx\_start\_address}}
The start address of this fixture, if it has been addressed. This is 
automatically assigned using the address function so shouldn't be changed 
manually. Used by: \texttt{plotter texlux}.

\paragraph{\texttt{universe}}
The universe in which the DMX channels for this fixture are located. This too 
is set when the address command is run so shouldn't be changed by the user. 
Used by: \texttt{plotter}.

\paragraph{\texttt{posX}}
The x position in 2D space where this fixture is located. Measured in metres.

\paragraph{\texttt{posY}}
The y position in 2D space where this fixture is located. Measured in metres.

\paragraph{\texttt{focusX}}
The x position in 2D space where the centre of this fixture's beam is 
focused. Measured in metres.

\paragraph{\texttt{focusY}}
The y position in 2D space where the centre of this fixture's beam is 
focused. Measured in metres.

\paragraph{\texttt{rotation}}
The rotation of this fixture about its centre. Measured anticlockwise from 
the positive x axis in radians. This can be automatically calculated if the 
preceding four data tags are present.

\paragraph{\texttt{dimmer}}
For traditional fixtures only. The dimmer that is controlling this fixture. 
Used by: \texttt{texlux}.

\paragraph{\texttt{circuit}}
For traditional fixtures only. The circuit into which the fixture is patched. 
Used by: \texttt{texlux}.

\paragraph{\texttt{power}}
For traditional fixtures only. The maximum power draw by the lamp in this 
fixture.

\paragraph{\texttt{gel}}
The name or manufacturer's code of the gel that is being used in this 
fixture.

\paragraph{\texttt{colour}}
The colour in which this fixture should be rendered on lighting plots. This 
can be calculated automatically if the preceding tag is present and is a 
standard colour name or code. (NYI)

\subsection{The Registries Section}
The registries section contains one or more DMX registries, each of which 
contains some channel elements, which contain both the UUID of the fixture 
they control and their function. There is very little that the user can do 
to directly control this section as all addresses are assigned based on 
the manipulation of fixtures.

\section{Using the Command Line Interface}
Pylux ships with two interface modes: a command line interface (CLI) and a 
graphical user interface (GUI). The interface that is loaded on launch is 
specified by the interface flag when the program is run. However, the 
default interface is the CLI. When the CLI launches, you are presented with an 
interactive prompt, which accepts commands which may also have arguments. Each 
command is two letters long (apart from the utility commands), where the first 
letter represents the aspect that the command affects and the second letter 
represents the action of the command.

When you are using the CLI, you will notice that some commands call for you 
to provide an interface reference id (usually called \texttt{\textit{REF}} 
in the command synopsis). This allows you to easily pass objects that have 
been listed on-screen into another command. You can tell which references are 
legal because they will be underlined. For example to pipe a fixture object 
into the \texttt{xs} command, you would first run a command which lists the 
fixture you wish to change (\texttt{xl} or \texttt{xf}), find the interface 
reference of that fixture, then use that reference in the \texttt{xs} 
command. In addition to these references, there is an additional special 
reference which is not displayed on-screen which is called using the 
reference \texttt{this}. This special reference will refer to the object that 
was previously worked on. For example, if you perform \texttt{xs} on a fixture 
then wish to perform another \texttt{xs} immediately afterwards, you can use 
\texttt{this} instead of the interface reference.

\subsection{Utility Commands}

\paragraph{\texttt{h}}
Display a list of the available commands for the interactive prompt. This 
prints the contents of \texttt{help.txt}  to the output.

\paragraph{\texttt{c}}
Clears the screen of previous input and output. This uses the system screen 
clearing command. (\texttt{clear} on UNIX, \texttt{cls} on Windows)

\paragraph{\texttt{q}}
Exit the program and autosave any changes that have been made to the tree.

\paragraph{\texttt{Q}}
Exit the program without saving any changes to disk.

\subsection{File Commands}

\paragraph{\texttt{fo \textit{FILE}}}
Open the file with path \texttt{\textit{FILE}} as the plot file. This will 
override any unsaved buffer associated with the previous plot file, if 
there was one.

\paragraph{\texttt{fw}}
Save the current file buffer to its original location.

\paragraph{\texttt{fW \textit{PATH}}}
Save the current file buffer to a new file with location 
\texttt{\textit{PATH}}.

\paragraph{\texttt{fg}}
Print the path of the file that is currently loaded.

\paragraph{\texttt{fn \textit{PATH}}}
Create an new empty plot file at the location with path \texttt{\textit{PATH}} 
and load it as the new plot file

\subsection{Metadata Commands}

\paragraph{\texttt{ml}}
List all the metadata tags associated with the currently loaded plot file.

\paragraph{\texttt{ms \textit{TAG VALUE}}}
Set the value of the metadata with tag \texttt{\textit{TAG}} to 
\texttt{\textit{VALUE}}. If the metadata already exists, it will be 
overridden.

\paragraph{\texttt{mr \textit{TAG}}}
Remove the piece of metadata from the file which has the name 
\texttt{\textit{TAG}}.

\paragraph{\texttt{mg \textit{TAG}}}
Get the value of a piece of metadata. Prints the value of the metadata with
name \texttt{\textit{TAG}} on the screen.

\subsection{Fixture Commands}

\paragraph{\texttt{xn \textit{OLID}}}
Add a new fixture to the plot. This will add a fixture to the plot with olid 
\texttt{\textit{OLID}}. It will also automatically assign a UUID and search 
the OLF file associated with this olid for any constants that this fixture has and add them to the fixture. This will not assign DMX addresses to the fixture,
use \texttt{xA} for that.

\paragraph{\texttt{xc \textit{REF}}}
Clone the fixture with interface reference \texttt{\textit{REF}} into a new 
fixture. This does not reassign any DMX values.

\paragraph{\texttt{xl}}
List all the fixtures in the plot. This will generate a list of every fixture 
in the plot, listing an interface reference, the fixture olid, and the fixture 
UUID.

\paragraph{\texttt{xf \textit{TAG VALUE}}}
List all the fixtures in the plot which have a tag called 
\texttt{\textit{TAG}} with a value of \texttt{\textit{VALUE}}. Like the list 
function, this will list an interface reference, the fixture olid and UUID, 
and also the value of the tag that was given for verification purposes.

\paragraph{\texttt{xg \textit{REF TAG}}}
Print the value of \texttt{\textit{TAG}} for the fixture with interface
reference \texttt{\textit{REF}}.

\paragraph{\texttt{xG \textit{REF}}}
List all the information associated with the fixture with interface reference 
\texttt{\textit{REF}}.

\paragraph{\texttt{xr \textit{REF}}}
Remove the fixture with the interface reference \texttt{\textit{REF}}. This 
fixture will be removed from the plot, but associated DMX channels will not be 
removed, use \texttt{xp} for that.

\paragraph{\texttt{xs \textit{REF TAG VALUE}}}
Set the value of \texttt{\textit{TAG}} in fixture with interface reference 
\texttt{\textit{REF}} to \texttt{\textit{VALUE}}.

\paragraph{\texttt{xA \textit{REF UNIVERSE ADDR}}}
Assign DMX addresses to the fixture with interface reference 
\texttt{\textit{REF}}. This will add the fixture to the universe 
\texttt{\textit{UNIVERSE}}, beginning at the start address 
\texttt{\textit{ADDR}}. \texttt{\textit{ADDR}} can either be a user-assigned 
number or auto to allow Pylux to choose the most appropriate start address.

\paragraph{\texttt{xp \textit{REF}}}
Remove the fixture with interface reference \texttt{\textit{REF}} from the 
plot and also remove any DMX channels associated with it.

\subsection{DMX Registry Commands}

\paragraph{\texttt{rl \textit{UNIVERSE}}}
List all the used channels in \texttt{\textit{UNIVERSE}}. This will list the 
DMX address, fixture UUID and function of every channel in the DMX registry 
with identifier \texttt{\textit{UNIVERSE}}.

\subsection{Using Extensions}
You cannot directly interact with extensions from the \texttt{plotter} 
interface, you must first load the extension using the \texttt{:} command.
For example, to load the \texttt{texlux} extension, use \texttt{:texlux}. 
This will then present you with the interface as defined by that extension 
which may vary but in practise should be a prompt of the form 
\texttt{pylux:extension} to indicate to you which extension you are 
operating in and some commands, much like in the \texttt{plotter} interface.
The extension defines its own way of returning to \texttt{plotter} but this 
should in general be \texttt{::} or \texttt{q}.

\section{Using the Graphical User Interface}
You may instead choose to launch Pylux using its GUI. This is NYI so please 
don't.

\section{Extensions}
In the previous sections, we have discussed the usage of the base program in 
Pylux: \texttt{plotter}. This is the program that you will use to edit your
plots, however, beyond that, it doesn't do much. That is why Pylux is also 
bundled with extensions to provide extra functionality. In the current 
version, Pylux comes bundled with the \texttt{texlux} extension only.

\subsection{\texttt{texlux}}
\texttt{texlux} is an extension to the base \texttt{plotter} program which allows for the creation of reports in the \LaTeX{} format, which can then be 
post-processed to create a PDF file, or many other formats through the use of 
an external tool such as Pandoc.

\subsubsection{Commands}

\paragraph{\texttt{rn \textit{TEMPLATE OUTPUT TITLE}}}
Generates a report using the template \texttt{\textit{TEMPLATE}}, with the 
title \texttt{\textit{TITLE}}, writing the output to a file with path 
\texttt{\textit{OUTPUT}}.

\subsubsection{Processing}

\texttt{texlux} uses built-in functions to generate \LaTeX{} docments with 
pre-defined structures. Each built-in function has a corresponding \LaTeX{} 
style file installed in \texttt{\~{}/.pylux/tex/} which is required to build the 
PDF report. Currently the only built-in function is \texttt{dimmer}, which 
produces a report categorised by dimmer and containing power draw totals.

\section{Developing for Pylux}
For the most part, Pylux is documented using Python docstrings, meaning you 
can access most documentation using the \texttt{pydoc} command. However, it 
is documented here anyway for the sake of legibility.

If you are developing an external program or interface and wish to use the 
Pylux system, Pylux provides an API to allow you to easily control the data of 
plot files in the form of the \texttt{plot} module. Once you have installed 
Pylux onto your system, you can then import this using 
\texttt{from pylux import plot}. This gives you access to a variety of 
objects to manipulate Pylux plot files.

Before each API function is documented, it is important to look at the 
structure of Pylux: an XML file, called a Pylux plot, is read and its data 
parsed using the \texttt{xml.etree.ElementTree} module. This means that the 
full XML tree is available in Python. However, Pylux does not directly edit 
this XML tree. Instead, Pylux creates empty Python objects then loads the 
data from the XML tree into these. This allows for easy editing and adding of 
information to these objects. Once these objects have been edited, they are 
then written back to the XML tree (usually using a \texttt{save()} function). 
Once Pylux is finished with a document, or when a user explicitly requests, 
it will then write the new tree back to the original XML file (or a new one if 
it has been specified). Your code should never directly alter or read from 
the XML tree. \texttt{plotter} doesn't even import the tree parser to ensure 
this. You should always access these objects via the API.

\subsection{The \texttt{PlotFile} Class}
This class deals with the Pylux plot file that is being used. There is no 
initialisation function.

\subsubsection{Functions of \texttt{PlotFile}}

\paragraph{\texttt{load(self, path)}}
Loads the file with location \texttt{path} as the working plot file. It will 
then parse the file into a Python-usable XML tree and store it in 
\texttt{tree}. It will then get the root element from the tree and store it in 
\texttt{root}.

\paragraph{\texttt{save()}}
Saves the current state of the XML working tree to the location from which 
the plot file was originally loaded.

\paragraph{\texttt{saveas(path)}}
Saves the current state of the XML working tree to the location with path 
\texttt{path}.

\paragraph{\texttt{generate(path)}}
Creates an empty plot file containing the \texttt{olplot}, \texttt{metadata} 
and \texttt{fixtures} tags at the location with path \texttt{path}. This does 
not load the file as the working file.

\subsubsection{Attributes of \texttt{PlotFile}}
Whilst there are attributes defined in the \texttt{PlotFile} class, in 
general they should not be used by extensions as this breaks down the 
tiered structure explained in the opening paragraph.

\paragraph{\texttt{file}}
The path of the plot file.

\paragraph{\texttt{tree}}
The Python XML tree element that was parsed from the file using 
\texttt{xml.etree.ElementTree}.

\paragraph{\texttt{root}}
The root element (\texttt{olplot}) of the plot file.

\subsection{The \texttt{DmxRegistry} Class}
The \texttt{DmxRegistry} class is Pylux's method for managing DMX registries. 
If you are using more than one DMX registry, each registry should call a new 
instance of \texttt{DmxRegistry}.

\subsubsection{Functions of \texttt{DmxRegistry}}

\paragraph{\texttt{\_\_init\_\_(plot\_file, universe)}}
This creates a new Python registry and sets its \texttt{universe} attribute 
to universe that was given as a parameter. It will then search the XML tree 
for a DMX registry with this universe id. If it finds one with the correct 
universe id, it will load the contents of that XML registry into the Python 
registry (a dictionary in the form \texttt{\{address: (uuid, func), ...\}}).
If it cannot find a registry with this universe id, it will instead create a 
new XML DMX registry object and add this to the tree.

\paragraph{\texttt{save()}}
This function simply saves the current state of the DmxRegistry object to the 
XML tree. 

\paragraph{\texttt{get\_occupied()}}
Returns a list of the occupied DMX channels in the Python registry.

\paragraph{\texttt{get\_start\_address(n)}}
Returns a recommeded start address for a fixture. This function will search 
through the registry to find the next \texttt{n} free DMX addresses in a row.

\paragraph{\texttt{address(fixture, start\_address)}}
Assigns DMX addresses to the fixture \texttt{fixture}. The function will start 
assigning at \texttt{start\_address} if it is an integer, or will call 
\texttt{get\_start\_address} if the string \texttt{auto} is given instead. 
The number of addresses to assign and the functions of these addresses is 
deterimned by accessing the fixture's attributes. Additionally, if the 
fixture has been previously addressed (whether it has or not is determined by 
getting the value of \texttt{universe} from its data), those channels will be 
removed before the new ones are assigned.

\paragraph{\texttt{unaddress(fixture)}}
Removes the DMX addresses from a fixture \texttt{fixture} and from the 
registry. Removes based on the fixture's \texttt{dmx\_channels} data tag.

\subsubsection{Attributes of \texttt{DmxRegistry}}

\paragraph{\texttt{plot\_file}}
The current plot file is stored as an attribute simply to make accessing it 
easier, it should not be used externally.

\paragraph{\texttt{registry}}
A dictionary containing all of the registry information, in the form 
\texttt{\{address: (uuid, func), ...\}}.

\paragraph{universe}
The name of the universe id of this registry.

\paragraph{xml\_registry}
This is another attribute which is present for internal puproses and should 
not be referenced by external programs; simply the XML data of the registry.

\subsection{The \texttt{FixtureList} Class}
The \texttt{FixtureList} class manages all the fixtures in a plot as a single 
entity to make filtering and listing easier. This is a small class present 
to make code cleaner.

\subsubsection{Functions of \texttt{FixtureList}}

\paragraph{\texttt{\_\_init\_\_(plot\_file)}}
Finds all the fixtures in XML and makes a \texttt{Fixture} object for them, 
then appends them to a list.

\paragraph{\texttt{remove(fixture)}}
Removes the fixture \texttt{fixture} from the plot.

\paragraph{\texttt{get\_data\_values(data\_type)}}
Returns a list of all the values that exist for \texttt{data\_type} in all the 
fixtures in the plot.

\subsubsection{Attributes of \texttt{FixtureList}}

\paragraph{\texttt{xml\_fixture\_list}}
The fixtures list in XML. Should not be referenced externally.

\paragraph{\texttt{fixtures}}
A list of \texttt{Fixture} objects, containing all the fixtures in the plot.

\subsection{The \texttt{Fixture} Class}
This class manages individual fixtures and their data.

\subsubsection{Functions of \texttt{Fixture}}

\paragraph{\texttt{\_\_init\_\_(plot\_file, uuid=None)}}
Creates a new \texttt{Fixture} object in Python and, if a fixture with the 
UUID \texttt{uuid} can be found in the XML file, loads the data of this 
fixture into the object.

\paragraph{\texttt{new(olid, fixtures\_dir)}}
Initialises this fixture as a new fixture. This will load information from a 
file located in \texttt{fixtures\_dir} with the name \texttt{olid.olf} and 
copy it into the \texttt{Fixture} object.

\paragraph{\texttt{add()}}
Adds the fixture, in its current state to the XML tree as a new fixture. In 
other words, it appends to the XML tree rather than checking if the fixture 
already exists.

\paragraph{\texttt{load(xml\_fixture)}}
Initialises this fixture as an existing fixture from the XML tree with 
XML fixture object \texttt{xml\_fixture}.

\paragraph{\texttt{clone(src\_fixture)}}
Clones the contents of \texttt{src\_fixture} into this fixture.

\paragraph{\texttt{save()}}
Saves the fixture to the XML tree. This will not add the fixture to the tree 
if it does not already exist in the tree. However, it will manage all other 
fixture data whether they are existing, new or removed.

\paragraph{\texttt{generate\_rotation()}}
Returns a value for rotation based on the fixture's position and focus 
position. These values must be defined in the fixture's data dictionary 
before this function is called.

\subsubsection{Attributes of \texttt{Fixture}}

\paragraph{plot\_file}
Like all other instances of \texttt{plot\_file} as class attributes, this 
should not be called externally.

\paragraph{data}
A dictionary containing all the child data of the fixture, except its DMX 
functions, in the form \texttt{\{data\_name: data\_value, ...\}}.

\paragraph{dmx\_functions}
A list of all the DMX functions that this fixture has.

\paragraph{olid}
This fixture's olid.

\paragraph{uuid}
This fixture's uuid.

\paragraph{xml\_fixture}
Another attribute which should not be referenced externally: the fixture in 
XML.

\paragraph{dmx\_num}
The number of DMX channels required by this fixture, calculated from 
\texttt{dmx\_functions}.

\subsection{The \texttt{Metadata} Class}
The \texttt{Metadata} class deals with the metadata section of the plot file.

\subsubsection{Functions of \texttt{Metadata}}

\paragraph{\texttt{\_\_init\_\_(plot\_file)}}
Loads the metadata from the XML tree into a Python dictionary.

\paragraph{\texttt{save()}}
Saves the current metadata dictionary to the XML tree.

\subsubsection{Attributes of \texttt{Metadata}}

\paragraph{xml\_meta}
The metadata in XML. Should not be externally referenced.

\paragraph{meta}
A dictionary containing all the metadata information in the form 
\texttt{\{meta\_name: meta\_value, ...\}}.

\subsection{Creating Fixture Symbols}
Pylux fixture symbols are written in SVG. A symbol SVG document should 
consist of the base SVG element, with an immediate child \texttt{g} element, 
with class \texttt{fixture}. This is the element that \texttt{plotter} will 
manipulate then insert into the plot. The fixture should be pointing towards 
the positive x direction, to allow proper rotation. Outer paths should have 
the class \texttt{outer}, these will be filled with the colour determined by 
the \texttt{colour} data tag. Inner paths should have the class 
\texttt{inner}, these will be filled white. Each coordinate point in SVG 
should represent one millimetre.

\end{document}
