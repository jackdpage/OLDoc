\documentclass[a4paper]{article}
\usepackage{hyperref}
\begin{document}
\title{Pylux User Manual}
\author{J. Page}
\date{2015}
\maketitle
\newpage
Copyright (C)  2015 Jack Page \\
Permission is granted to copy, distribute and/or modify this document
under the terms of the GNU Free Documentation License, Version 1.3
or any later version published by the Free Software Foundation;
with no Invariant Sections, no Front-Cover Texts, and no Back-Cover Texts.
A copy of the license is included in the section entitled "GNU
Free Documentation License".
\newpage
\tableofcontents
\newpage
\section{Introduction}
Pylux is a program written in Python for the manipulation of OpenLighting Plot
documents. Pylux currently allows for the basic editing of the OpenLighting
Plot XML files, including referencing OpenLighting Fixture files to obtain
additional data.

In the future, Pylux will be extended with modules that allow for the exporting
of documentation in the \LaTeX{} format, and creating lighting plots in SVG 
format.

Bugs and feature requests should be submitted to 
\url{https://github.com/jackdpage/pylux/issues}.

\section{Command-line Options}
Pylux is invoked simply by running the \texttt{plotter.py} program. Windows 
operating systems should automatically associate file with the Python 
interpreter. The program contains a shebang line for compatibility with UNIX
systems. 

\paragraph{\texttt{--help}, \texttt{-h}}
Display the usage message of the program then exit.

\paragraph{\texttt{--version}, \texttt{-v}}
Display the version number of the program then exit.

\paragraph{\texttt{--file \textit{FILE}}, \texttt{-f \textit{FILE}}} 
Load the file with path \textit{FILE} as the project file on launch.

\section{Basic Concepts}
Pylux uses the OpenLighting Plot format as its document format. This format 
specifies three main components of an XML document: the meta section, the 
fixtures section and the registries section. In its current version, Pylux 
can only read and write to the latter two sections.

The fixtures section contains multiple fixture elements, each of which has 
various data associated with it, depending on the fields as defined by the 
fixture's OpenLighting ID (olid).

The registries section contains one or more DMX registries, each of which 
contains some channel elements, which contain both the UUID of the fixture 
they control and their function.

\section{Fundamental Commands}
When launching Pylux, you are presented with an interactive prompt. This 
prompt accepts many commands, all of which are one or two characters long and 
may take one or more arguments. A complete list of commands is included below: 

\subsection{Utility Commands}

\paragraph{\texttt{h}}
Display a list of the available commands for the interactive prompt. This 
prints the contents of \texttt{help.txt}  to the output.

\paragraph{\texttt{c}}
Clears the screen of previous input and output. This uses the system screen 
clearing command. (\texttt{clear} on UNIX, \texttt{cls} on Windows)

\paragraph{\texttt{q}}
Exit the program and autosave any changes that have been made to the tree.

\paragraph{\texttt{q!}}
Exit the program without saving any changes to disk.

\subsection{File Commands}

\paragraph{\texttt{fl \textit{FILE}}}
Load the file with path \textit{FILE} as the project file. This will override 
any unsaved buffer associated with the previous project file, if there was one.

\paragraph{\texttt{fs}}
Save the current file buffer to its original location.

\paragraph{\texttt{fS \textit{PATH}}}
Save the current file buffer to a new file with location \textit{PATH}.

\subsection{Metadata Commands}

\paragraph{\texttt{ml}}
List all the metadata tags associated with the currently loaded project
file.

\paragraph{\texttt{ma \textit{TAG VALUE}}}
Add a new piece of metadata. Adds a new tag with the name \textit{TAG}
to the metadata section and sets its value to \textit{VALUE}.

\paragraph{\texttt{mr \textit{TAG}}}
Remove the piece of metadata from the file which has the name \textit{TAG}.

\paragraph{\texttt{mg \textit{TAG}}}
Get the value of a piece of metadata. Prints the value of the metadata with
name \textit{TAG} on the screen.

\subsection{Fixture Commands}

\paragraph{\texttt{xa \textit{OLID}}}
Add a new fixture to the plot. This will add a fixture to the plot with olid 
\textit{OLID}. It will also automatically assign a UUID and search the OLF 
file associated with this olid for any constants that this fixture has and 
add them to the fixture. This will not assign DMX addresses to the fixture,
use \texttt{xA} for that.

\paragraph{\texttt{xl}}
List all the fixtures in the plot. This will generate a list of every fixture 
in the plot, listing an interface reference, the fixture olid, and the fixture 
UUID.

\paragraph{\texttt{xf \textit{TAG VALUE}}}
List all the fixtures in the plot which have a tag called \textit{TAG} with a 
value of \textit{VALUE}. Like the list function, this will list an interface 
reference, the fixture olid and UUID, and also the value of the tag that was 
given for verification purposes.

\paragraph{\texttt{xi \textit{REF}}}
List all the information associated with the fixture with interface reference 
\textit{REF}. \textit{REF} is the number given to a fixture by either the 
list or filter commmand. It must be the number given by the most recently run 
command, as this interface reference buffer is updated each time one of these 
commands is run.

\paragraph{\texttt{xr \textit{REF}}}
Remove the fixture with the interface reference \textit{REF}. This fixture 
will be removed from the plot, but associated DMX channels will not be 
removed, use the upcoming command \texttt{xp} for that.

\paragraph{\texttt{xs \textit{REF TAG VALUE}}}
Set the value of \textit{TAG} in fixture with interface reference \textit{REF} 
to \textit{VALUE}.

\paragraph{\texttt{xA \textit{REF UNIVERSE ADDR}}}
Assign DMX addresses to the fixture with interface reference \textit{REF}.
This will add the fixture to the universe \textit{UNIVERSE}, beginning at the 
start address \textit{ADDR}. \textit{ADDR} can either be a user-assigned 
number or auto to allow Pylux to choose the most appropriate start address.

\subsection{DMX Registry Commands}

\paragraph{\texttt{rl \textit{UNIVERSE}}}
List all the used channels in \textit{UNIVERSE}. This will list the DMX 
address, fixture UUID and function of every channel in the DMX registry with 
identifier \textit{UNIVERSE}.

\section{Extensions}
In the previous sections, we have discussed the usage of the base program in 
Pylux: \texttt{plotter}. This is the program that you will use to edit your
plots, however, beyond that, it doesn't do much. That is why Pylux can be 
extended with additional Python scripts to provide extra functionality. In 
the current version, Pylux comes bundled with the \texttt{genlux} script only.

\subsection{genlux}
genlux is an extension to the base \texttt{plotter} program which allows for 
the creation of reports in the \LaTeX{} format, which can then be 
post-processed to create a PDF file, or many other formats through the use of 
an external tool such as Pandoc.

In its current form, genlux can only create dimmer reports. This will change 
in the near future. As such, genlux has very little in the way of options. As 
genlux imports \texttt{plotter} when it is run, it accepts the same command 
line arguments as \texttt{plotter}. Hence genlux is invoked with 
\texttt{genlux -f \textit{FILE}} where \textit{FILE} is a Pylux plot file.
This will generate a \LaTeX{} source in STDOUT, which can be written to a 
file or piped directly into a \LaTeX{} processing tool such as 
\texttt{pdflatex}.

\end{document}
